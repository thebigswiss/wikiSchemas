%%%%%%%%%%%%%%%%%%%%%%%%%%%%%%%%%%%%%%%%%%%%%%%%%%%%%%%%%
%
% CBlindleistung
%
% Author: Ambros Ellenberger (thebigswiss)
% Version: see on https://github.com/thebigswiss/wikiSchemas
% License: GPLv3, see GitHub
% https://de.wikipedia.org/wiki/Kondensator_(Elektrotechnik)#/media/Datei:Blindleistung.svg
% https://de.wikipedia.org/wiki/Kondensator_(Elektrotechnik)#/media/Datei:Blindleistungskompensation.svg
%
%%%%%%%%%%%%%%%%%%%%%%%%%%%%%%%%%%%%%%%%%%%%%%%%%%%%%%%%%
%
\documentclass[landscape,12pt]{article}
%
%
\usepackage{siunitx}
\sisetup{per-mode=fraction}
%
\usepackage{pgfplots}
\pgfplotsset{compat=1.15}
%
\usepackage{mathrsfs}
%
\usepackage{tikz}
\usepackage[european, straightvoltages, americaninductors, siunitx]{circuitikz}
%
\ctikzset{logic ports=european, tripoles/european not symbol=ieee circle}
\usetikzlibrary{arrows}
%
\pagestyle{empty}
\begin{document}
	
\begin{circuitikz}
	
	\draw (0,0) coordinate(Ue);
	\draw (Ue) ++ (2,0) coordinate(L11);
	\draw (L11) -- ++ (3,0) coordinate(C11);
	\draw (C11) -- ++ (2,0) coordinate(R11);
	\draw (R11) ++ (0,-3) coordinate(R12);
	\draw (R12) -- ++ (-2,0) coordinate(C12);
	\draw (C12) -- ++ (-3,0) coordinate(L12);
	\draw (L12) -- ++ (-2,0) coordinate(GND);
	
	\draw (Ue) to[open, v = \SI{230}{\volt}, o-o] (GND);
	\draw (L11) to[L, l = \SI{500}{\milli\henry}, i = \SI{1.45}{\ampere}, *-*] (L12);
	\draw (C11) to[C, l = \SI{20}{\micro\farad}, i = \SI{1.45}{\ampere}, *-*] (C12);
	\draw (R11) to[R, l = \SI{100}{\ohm}] (R12);
	
	\draw (Ue) to[short, i = \SI{2.3}{\ampere}] (L11);
	\draw (L11) to[short, i = \SI{2.72}{\ampere}] (C11);
	\draw (C11) to[short, i = \SI{2,3}{\ampere}] (R11);
	
	%% Beschriftungen
	
	\draw (C11) ++ (-0.5, 0.5) coordinate(edge1);
	\draw (R12) ++ (0.5, -1) coordinate(edge2);
	
	\draw[red, thick] (L12) node[below]{Kompensation};
	
	\draw[red, thick] (edge1) rectangle (edge2) node[anchor=south east, align=center] at (body.south east) {Ding};
	


\end{circuitikz}
	
\end{document}