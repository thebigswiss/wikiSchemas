%%%%%%%%%%%%%%%%%%%%%%%%%%%%%%%%%%%%%%%%%%%%%%%%%%%%%%%%%
%
% Template for creating classic Schmeas, with circuitikz.
%
% Author: Ambros Ellenberger (thebigswiss)
% Version: see on https://github.com/thebigswiss/wikiSchemas
% License: GPLv3, see GitHub
%
%%%%%%%%%%%%%%%%%%%%%%%%%%%%%%%%%%%%%%%%%%%%%%%%%%%%%%%%%
%
\documentclass[landscape]{article}
%
%
\usepackage{siunitx}
\sisetup{per-mode=fraction}
%
\usepackage{pgfplots}
\pgfplotsset{compat=1.15}
%
\usepackage{mathrsfs}
%
\usepackage{tikz}
\usepackage[european, straightvoltages, americaninductors]{circuitikz}
%
\ctikzset{logic ports=european, tripoles/european not symbol=ieee circle}
\usetikzlibrary{arrows}
%
\pagestyle{empty}
\begin{document}
	
	\begin{circuitikz}
		
		\ctikzset{diodes/scale=0.6}
		
		\draw(0,0) coordinate(Ue1);
		\draw(0,0) --++ (1,0) coordinate(C1);
		\draw(C1) ++ (0,-3) coordinate(C2);
		\draw(C2) --++ (-1,0) coordinate(Ue2);
		\draw(C1) --++ (1,0) coordinate(MOSD);
		\draw(MOSD) ++ (2,0) coordinate(MOSS);
		\draw(MOSS) --++ (1,0) coordinate(D1);
		\draw(D1) ++ (0,-3) coordinate(D2);
		\draw(D2) -- (C2);
		\draw(D1) --++ (1,0) coordinate(L1);
		\draw(L1) ++ (2,0) coordinate(L2);
		\draw(L2) --++ (1,0) coordinate(C3);
		\draw(C3) ++ (0,-3) coordinate(C4);
		\draw(C4) -- (D2);
		\draw(C3) --++ (1,0) coordinate(Ua1);
		\draw(C4) --++ (1,0) coordinate(Ua2);
		
		\draw(MOSS) to[Tnigfete] (MOSD);
		\draw(C1) to[eC, l=$C_1$,*-*] (C2);
		\draw(Ue1) to[open, v = $U_e$, o-o] (Ue2);
		\draw(D2) to[D,l=$D_1$, *-*] (D1);
		\draw(L1) to[L, l=$L_1$] (L2);
		\draw(C3) to[eC,l_=$C_2$, *-*] (C4);
		\draw(Ua1) to[open, v^>= $U_a$, o-o] (Ua2);
	
	\end{circuitikz}
	
\end{document}